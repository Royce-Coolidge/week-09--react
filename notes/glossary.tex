\begin{itemize}[leftmargin=*]
    \item
        \textbf{Component}:
        a reusable module representing some small part of a user interface
    \item
        \textbf{Controlled Component}:
        a form input whose value is controlled by the component's state
    \item
        \textbf{JSX}:
        React's templating language. Converts a mix of HTML and JavaScript into HTML
    \item
        \textbf{Lifting State}:
        taking state out of a child component and putting it in the parent component
    \item
        \textbf{Moustaches}:
        a colloquial name for \texttt{\{} and \texttt{\}} when used in templating languages
    \item
        \textbf{Props}:
        props are written like HTML attributes and allow us to pass values into sub-components
    \item
        \textbf{Router}:
        some code that decides what to do depending on the provided URL
    \item
        \textbf{Route}:
        a piece of code to run given a specific URL
    \item
        \textbf{Shape}:
        the structure of your state
    \item
        \textbf{Stateful Component}:
        a \texttt{class} with a \texttt{render} method and the \texttt{this.props} and \texttt{this.state} properties
    \item
        \textbf{Stateless Component}:
        a function that accepts props and returns JSX
    \item
        \textbf{State}:
        keeping track of changes using long-lived variables
    \item
        \textbf{Sub-Component}:
        a component that is used by another component
    \item
        \textbf{Templating Language}:
        converts a mixture of markup and a programming language to create pure markup
    \item
        \textbf{URL Parameter}:
        part of a URL that can be used like a variable
\end{itemize}
